\documentclass[a4paper,14pt]{extarticle}

\usepackage{cmap}
\usepackage[T2A]{fontenc}
\usepackage[utf8x]{inputenc}
\usepackage[english, russian]{babel}

\usepackage{misccorr}
\usepackage{amssymb,amsfonts,amsmath,amsthm}  
\usepackage{indentfirst}
\usepackage[usenames,dvipsnames]{color} 
\usepackage[unicode,hidelinks]{hyperref}
\usepackage{makecell,multirow} 
\usepackage{ulem}
\usepackage{graphicx,wrapfig}
\graphicspath{{img/}}

\renewcommand{\labelenumii}{\theenumii)} 
\newcommand{\mean}[1]{\langle#1\rangle}

\DeclareMathOperator{\Div}{div}
\DeclareMathOperator{\const}{const}
%%%%%%%%%%%%%%%%%%%%%%%%%%%%%%%%%%%%%%%%%%%%%%%%%%%%%%%%%%%%%%%%%%%%%%%%%%%%%%%
%%%%%%%%%%%%%%%%%%%%%%%%%%%%%%%%%%%%%%%%%%%%%%%%%%%%%%%%%%%%%%%%%%%%%%%%%%%%%%%
\usepackage{float}
\usepackage[mode=buildnew]{standalone}
\usepackage[outline]{contour}
\usepackage{tocloft}
\renewcommand{\cftsecleader}{\cftdotfill{\cftdotsep}} % for parts
% \renewcommand{\cftchapleader}{\cftdotfill{\cftdotsep}} % for chapters
\usepackage{pgfplots,pgfplotstable,booktabs,colortbl}
\usepackage{physics}
\usepackage{mathtools}
\mathtoolsset{showonlyrefs=true}

\newcommand*\dotvec[1][1,1]{\crossproducttemp#1\relax}
\def\crossproducttemp#1,#2\relax{{\qty[\vec{#1}\times\vec{#2}\,]}}

\newcommand*\prodvec[1][1,1]{\crossproducttempa#1\relax}
\def\crossproducttempa#1,#2\relax{{\qty[{#1}\times{#2}\,]}}
% \usepackage{showframe}
\usepackage[]{geometry}
\geometry{
  left=2.5cm,
  right=1.5cm,
  top=2cm,
  bottom=2cm,
  bindingoffset=0cm,
  headheight=17pt
}
\linespread{1.5} 
\setlength{\parindent}{1.25cm}
\frenchspacing 
\usepackage{setspace}

\begin{document}

\begin{titlepage}
  \begin{center}
    {\fontsize{ 12pt }{ 12pt } \selectfont \bf 
    МИНИСТЕРСТВО НАУКИ И ВЫСШЕГО ОБРАЗОВАНИЯ \\[-10pt] 
    РОССИЙСКОЙ ФЕДЕРАЦИИ}\\
    \vspace{12pt}
    \begin{spacing}{1}
      {\bf  Федеральное государственное автономное \\
      образовательное учреждение высшего образования \\
      <<Национальный исследовательский Нижегородский \\ 
      государственный университет им. Н.И. Лобачевского>>
      }
    \end{spacing}
    \vspace{24pt}
    \begin{spacing}{1}
      Радиофизический факультет\\
      Кафедра общей физики\\
      \vspace{20pt}
      Направление <<Фундаментальная радиофизика>>\\
      \vspace{20pt}
      ОТЧЕТ ПО $\ldots$ ПРАКТИКЕ
    \end{spacing}
    \vspace{100pt}
    \begin{equation}
      \begin{aligned}
        &\text{Руководитель практики:}\quad &\text{Мареев Е.\,А.}\\
        &\text{Студент 3-го курса бакалавриата:}\quad &\text{Сарафанов Ф.\,Г.}
      \end{aligned}
    \end{equation}
  \end{center}
  \vfill
  \begin{center}
    {Нижний Новгород, 2019}
  \end{center}
\end{titlepage}

\tableofcontents
\newpage

Lorem ipsum dolor sit amet, consectetur adipisicing elit, sed do eiusmod
tempor incididunt ut labore et dolore magna aliqua. Ut enim ad minim veniam,
quis nostrud exercitation ullamco laboris nisi ut aliquip ex ea commodo
consequat. Duis aute irure dolor in reprehenderit in voluptate velit esse
cillum dolore eu fugiat nulla pariatur. Excepteur sint occaecat cupidatat non
proident, sunt in culpa qui officia deserunt mollit anim id est laborum.

Lorem ipsum dolor sit amet, consectetur adipisicing elit, sed do eiusmod
tempor incididunt ut labore et dolore magna aliqua. Ut enim ad minim veniam,
quis nostrud exercitation ullamco laboris nisi ut aliquip ex ea commodo
consequat. Duis aute irure dolor in reprehenderit in voluptate velit esse
cillum dolore eu fugiat nulla pariatur. Excepteur sint occaecat cupidatat non
proident, sunt in culpa qui officia deserunt mollit anim id est laborum.


\section*{Введение}
\addcontentsline{toc}{section}{Введение}

Lorem ipsum dolor sit amet, consectetur adipisicing elit, sed do eiusmod
tempor incididunt ut labore et dolore magna aliqua. Ut enim ad minim veniam,
quis nostrud exercitation ullamco laboris nisi ut aliquip ex ea commodo
consequat. Duis aute irure dolor in reprehenderit in voluptate velit esse
cillum dolore eu fugiat nulla pariatur. Excepteur sint occaecat cupidatat non
proident, sunt in culpa qui officia deserunt mollit anim id est laborum.

\section{Теоретическая часть}

\section{Экспериментальная установка}

Аппаратура, размещенная на территории пункта Городец, позволяла вести наблюдения электрического поля, тока и проводимости как в невозмущенной атмосфере, так и в грозовых условиях, включая быстрые изменения во время разрядов молний. Флюксметр регистрировал значения электрического поля в полосе частот 0--10 Гц. Для наблюдения сигналов ближних гроз в СДВ диапазоне (500-10000 Гц) была использована малогабаритная трехкомпонентная приемная система (две компоненты горизонтального магнитного поля и одна компонента вертикального электрического поля). 



Сигналы с приемной системы поступали на АЦП и после дискретизации записывались на накопитель. Установка работала в непрерывном режиме, генерируя поток данных около $700$ Мб/ч. Чтобы увеличить время непрерывной записи до заполнения накопителя, АЦП работал в пороговом режиме. 

На предшествующих измерениях была определена средняя амплитуда электрического поля регистрируемого молниевого разряда при грозе на расстоянии 100 км от датчиков. При амплитуде ниже пороговой АЦП работал на частоте дискретизации $f_\text{slow}=20$ Гц (медленный режим); при превышении амплитудой порога  АЦП переключался на частоту $f_\text{fast}=20$ кГц (быстрый режим). Спустя час быстрого режима, если отсутствовали импульсы выше порогового, АЦП переходил в медленный режим.


\addcontentsline{toc}{section}{Список литературы}
\begin{thebibliography}{99}
\bibitem{land} Ландау Л.Д., Лифшиц Е.М. Теоретическая физика: т.5, Статистическая физика. - М.: Физматлит, 2005. - 616 с.
\end{thebibliography}

\end{document}